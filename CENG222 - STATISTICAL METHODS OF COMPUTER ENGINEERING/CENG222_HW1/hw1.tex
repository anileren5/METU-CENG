\documentclass[11pt]{article}
\usepackage[utf8]{inputenc}
\usepackage[dvips]{graphicx}
\usepackage{fancybox}
\usepackage{verbatim}
\usepackage{array}
\usepackage{latexsym}
\usepackage{alltt}
\usepackage{hyperref}
\usepackage{textcomp}
\usepackage{color}
\usepackage{amsmath}
\usepackage{amsfonts}
\usepackage{tabularx}
\usepackage{tikz}
\usepackage{fitch}  % to use fitch
\usepackage{float}
\usepackage[hmargin=3cm,vmargin=5.0cm]{geometry}
%\topmargin=0cm
\topmargin=-2cm
\addtolength{\textheight}{6.5cm}
\addtolength{\textwidth}{2.0cm}
%\setlength{\leftmargin}{-5cm}
\setlength{\oddsidemargin}{0.0cm}
\setlength{\evensidemargin}{0.0cm}


\begin{document}

\section*{Student Information } 
%Write your full name and id number between the colon and newline
%Put one empty space character after colon and before newline
Full Name :  Anıl Eren Göçer \\
Id Number :  2448397 \\

\section*{Answer 1} 
BOX 1: 2 White and 8 Black \\
BOX 2: 4 White and 11 Black \\
BOX 3: 3 White and 9 Black \\


\section*{a)}
\textbf{By Complement Rule}; \\ \\ 
P\{At least one of them is white.\} = 1 - P\{None of them is white.\} = 1 - P\{All of them are black.\} \\  \\
= 1 - (\dfrac{8}{10}).(\dfrac{11}{15}).(\dfrac{9}{12}) = 1 - \dfrac{792}{1800} = \dfrac{1008}{1800} = 0.56 \\ \\
Answer = \textbf{0.56}


\section*{b)}
P\{All of the three balls drawn are white.\} = (\dfrac{2}{10}).(\dfrac{4}{15}).(\dfrac{3}{12}) = \dfrac{24}{1800} = 0.133333 \\ \\
Answer = \textbf{0.133333}

\section*{c)}
If I draw two balls from the, \\ \\

\textbf{BOX 1:} P\{get two white balls\} = (\dfrac{2}{10}).(\dfrac{1}{9}) = \dfrac{2}{90} = 0.022222 \\ \\ 

\textbf{BOX 2:} P\{get two white balls\} = (\dfrac{4}{15}).(\dfrac{3}{14}) = \dfrac{12}{210} = 0.05714 \\ \\ 

\textbf{BOX 3:} P\{get two white balls\} = (\dfrac{3}{12}).(\dfrac{2}{11}) = \dfrac{10}{132} = 0.04545 \\ \\ 

\noindent As seen above, probability of getting two white balls is highest in the case I draw from the \textbf{BOX 2}, since 0.05714 $>$ 0.04545 $>$ 0.022222. \\
Hence, I would draw from \textbf{BOX 2}. \\ \\
Answer = \textbf{BOX 2} \newpage


\section*{d)}
If I first draw from the \textbf{BOX 1} and then again from \textbf{BOX 1}: \space
P = (\dfrac{2}{10}).(\dfrac{1}{9}) = \dfrac{2}{90} = 0.022222 \\ \\

\noindent If I first draw from the \textbf{BOX 1} and then from \textbf{BOX 2}: \space
P = (\dfrac{2}{10}).(\dfrac{4}{15}) = \dfrac{8}{150} = 0.053333 \\ \\

\noindent If I first draw from the \textbf{BOX 1} and then from \textbf{BOX 3}: \space
P = (\dfrac{2}{10}).(\dfrac{3}{12}) = \dfrac{6}{120} = 0.05 \\ \\  \\

\noindent If I first draw from the \textbf{BOX 2} and then from \textbf{BOX 1}: \space
P = (\dfrac{4}{15}).(\dfrac{2}{10}) = \dfrac{8}{150} = 0.053333 \\ \\

\noindent If I first draw from the \textbf{BOX 2} and then again from \textbf{BOX 2}: \space
P = (\dfrac{4}{15}).(\dfrac{3}{14}) = \dfrac{12}{210} = 0.05714 \\ \\

\noindent If I first draw from the \textbf{BOX 2} and then from \textbf{BOX 3}: \space
P = (\dfrac{4}{15}).(\dfrac{3}{12}) = \dfrac{12}{180} = 0.066667 \\ \\ \\

\noindent If I first draw from the \textbf{BOX 3} and then from \textbf{BOX 1}: \space
P = (\dfrac{3}{12}).(\dfrac{2}{10}) = \dfrac{6}{120} = 0.05 \\ \\

\noindent If I first draw from the \textbf{BOX 3} and then from \textbf{BOX 2}: \space
P = (\dfrac{3}{12}).(\dfrac{4}{15}) = \dfrac{12}{180} = 0.066667 \\ \\

\noindent If I first draw from the \textbf{BOX 3} and then again from \textbf{BOX 3}: \space
P = (\dfrac{3}{12}).(\dfrac{2}{11}) = \dfrac{6}{132} = 0.04545 \\ \\ \\ 

So, probability of getting two white balls is the highest in the case I draw one
ball from \textbf{BOX 2} and one ball from \textbf{BOX 3}. \\

\textbf{As seen above, order does not matter. } \\ \\
Hence, First I would draw one ball from \textbf{BOX 2} and then draw one ball from
\textbf{BOX 3}, or \textbf{vice versa} because order is not important.  \\ \\

\noindent Answer: (\textbf{BOX 2}, \textbf{BOX 3}) or (\textbf{BOX 3}, \textbf{BOX 2})
\newpage

\section*{e)}
Let random variable X be the number of white balls. \\ \\
I will list the possibilities in such an order: \\
For example; \\ \\
WWB $\longrightarrow$ Write from BOX 1, Write from BOX 2, Black from BOX 3 \\ \\
Support of X = \{0,1,2,3\} \\ \\

\begin{center}
    The Distribution of X
\end{center} \\ 

P\{X = 0\} = P\{BBB\} = (\dfrac{8}{10}).(\dfrac{11}{15}).(\dfrac{9}{12}) = \dfrac{792}{1800} \\ \\

P\{X = 1\} = P\{WBB $\cup$  BWB $\cup$ BBW\} = (\dfrac{2}{10}).(\dfrac{11}{15}).(\dfrac{9}{12})+
(\dfrac{8}{10}).(\dfrac{4}{15}).(\dfrac{9}{12})+
(\dfrac{8}{10}).(\dfrac{11}{15}).(\dfrac{3}{12}) = \dfrac{750}{1800} \\ \\

P\{X = 2\} = P\{BWW $\cup$ WBW $\cup$ WWB\} = 
(\dfrac{8}{10}).(\dfrac{4}{15}).(\dfrac{3}{12})+
(\dfrac{2}{10}).(\dfrac{11}{15}).(\dfrac{3}{12})+
(\dfrac{2}{10}).(\dfrac{4}{15}).(\dfrac{9}{12}) = \dfrac{234}{1800} \\ \\

P\{X = 3\} = P\{WWW\} = (\dfrac{2}{10}).(\dfrac{4}{15}).(\dfrac{3}{12}) = \dfrac{24}{1800} \\ \\

\noindent Now, I will calculate expected value of X: \\ \\

E(X) = (\dfrac{792}{1800}).0 (\dfrac{750}{1800}).1 + (\dfrac{234}{1800}).2 +
(\dfrac{24}{1800}).3 = \dfrac{1290}{1800} = 0.716667 \\ \\ \\

\noindent Answer = \textbf{0.716667} \newpage

\section*{f)}
We will compute a conditional probability: \\ \\

Because we know that we get a white ball, total number of outcomes is equal to
2 + 4 / 3 = 9. \\ \\

Thw white ball we get can come from one of 2 white balls in BOX 1, so the number of
favorable outcomes is equal to 2.  \\ \\

\begin{center}
    P &= $\dfrac{N_{Favorable}}{N_{Total}}$ = $\dfrac{2}{9}$
\end{center}

Hence, probability that this ball was taken from \textbf{BOX 1} is \textbf{\dfrac{2}{9}}. \\ \\ 


\noindent Answer = \textbf{\dfrac{2}{9}}



\newpage
\section*{Answer 2}
\section*{a)}
S = \{Sam is corrupted.\} \\
D = \{The ring is destroyed.\}
\\ \\ It is given that, \\ \\ 
\textbf{i)} P\{D $|$ \overline{S}\} = 0.9 \newline
\textbf{ii)}P\{D | S\} = 0.5 \newline
\textbf{iii)} P\{S\} = 0.1 \newline \newline 
\textbf{By applying Complement Rule to iii)} \longrightarrow P\{\overline{S}\} = 1 - P\{S\} = 1 - 0.1 = 0.9 \\

\noindent I am required to find P\{S $|$ D\}, which is the probability that Sam is corrupted given that the ring is destroyed. \newline

\noindent Apply \textbf{Bayes Rule} and \textbf{Law of Total Probability}, \newline \newline
\noindent
P\{S $|$ D\} = \dfrac{P\{D | S\} . P\{S\}}{P\{D\}} = \dfrac{P\{D | S\} . P\{S\}}{P\{D | S)\} . P\{S\} + P\{D | \overline{S}\} . P\{\overline{S}\}} = \dfrac{(0.5) . (0.1)}{(0.5).(0.1) + (0.9).(0.9)} = 0.05813953488 \newline \newline
\noindent Answer = \textbf{0.05813953488}

\section*{b)}
F = {Frodo is corrupted.} \\ \\ 
It is given that, \\ \\
\textbf{i)} P\{F\} = 0.25 \\
\textbf{ii)} P\{D $|$ F\} = 0.2 \\
\textbf{iii)} P\{D $|$ $\overline{S} \cap \overline{F}$\} = 0.9 \\
\textbf{iv)} P\{D $|$ S \cap F \}  = 0.5 \\

\noindent Because the corruption of Frodo and Sam are \textbf{independent events}, 
$P\{S \cap F \}$ = P\{S\} . P\{F\} = (0.1) . (0.25) = 0.025 \\ 

\noindent \textbf{By applying Complement Rule}, \\ \\
P\{\overline{S \cap F}\} = 1 - P\{S \cap F\} = 1 - 0.025 = 0.975 \\
P\{D | \overline{S \cap F \}} = 1 - P\{D | S \cap F \} = 1 - 0.05 = 0.95
\newpage

\noindent I am required to find $P\{S \cap F | D\}$ , which is the probability that both Frodo and Sam are corrupted given the ring is destroyed. \\ \\
\noindent Apply \textbf{Bayes Rule} and \textbf{Law of Total Probability}, \newline \newline

\noindent $P\{S \cap F | D\}$ = \dfrac{P\{D | S \cap F\} . P\{S \cap F \}}{P\{D\}}
\dfrac{P\{D | S \cap F\} . P\{S \cap F \}}{P\{D | S \cap F\} . P\{S \cap F \} + 
P\{D | \overline{S \cap F} \}.P\{\overline{S\cap F}\}} \\ = \dfrac{(0.05).(0.025)}{
(0.05).(0.025) + (0.95).(0.075)} \newline \newline  = 0.0013477 \\ \\

\noindent Answer = \textbf{0.0013477}



\newpage















\section*{Answer 3}
\section*{a)}
A = \{The number of snowy days in Ankara \} \\
I = \{The number of snowy days in Istanbul \} \\

\noindent It is given that in the table; \\
Support of A = \{1,2,3\} \\ 
Support of B = \{1,2\} \\ \\

\noindent The event "there are 4 snowy days in total" consists of \\ 
- "2 snowy days in Ankara and 2 snowy days in Istanbul" ,\\
- "3 snowy days in Ankara and 1 snowy days in Istanbul" \\ \\ 

\noindent P\{Four snowy days in total\} = P\{a = 2, i=2\} + P\{a = 3, i = 1\} = 0.2 + 0.12 = 0.32 \\ \\
\noindent Hence, the probability that there are four snowy days in total is \textbf{0.32} \\ \\
Answer = \textbf{0.32}

\section*{b)}
For snowy days in Ankara and Istanbul to be independent, \\ \\
P\{A = a, I = i\} =  P\{A = a\} . P\{I = i\} \space \space \textbf{(1)}\\ must be satisfied for all a and i in supports of A and I, respectively.  If it is not satisfied, then this means they are dependent. \\ \\
\noindent
I will calculate marginal probabilities by using \textbf{Addition Rule} given in the textbook.
\newline

\noindent P\{A = 1\} = P\{A = 1,I = 1\} + P\{A = 1, I = 2\} = 0.18 + 0.12 = 0.30 \\
P\{A = 2\} = P\{A = 2, I = 1\} + P\{A = 2, I = 2\} = 0.30 + 0.20 = 0.50 \\
P\{A = 3\} = P\{A = 3, I = 1\} + P\{A = 3, I = 2\} = 0.12 + 0.08 = 0.20 \\

\noindent P\{I = 1\} = P\{A = 1, I = 1\} + P\{A = 2, I = 1\} + P\{A = 3, I = 1\}  = 0.18 + 0.30 + 0.12 = 0.60 \\
\noindent P\{I = 2\} = P\{A = 1, I = 2\} + P\{A = 2, I = 2\} + P\{A = 3, I = 2\}  = 0.12 + 0.2 + 0.08 = 0.40 \\ \\
We have obtained, \\ \\
\begin{tabularx}{0.4\textwidth}{ 
  | >{\raggedright\arraybackslash}X 
  | >{\centering\arraybackslash}X 
  | >{\raggedleft\arraybackslash}X | }
 \hline
  \textbf{a} & \textbf{P\{A = a\}}\\
 \hline
 1  & 0.3\\
  \hline
 2  & 0.5\\
  \hline
 3  & 0.2\\
\hline \space \space \space
\end{tabularx} \space \space \space \space 
\begin{tabularx}{0.4\textwidth} { 
  | >{\raggedright\arraybackslash}X 
  | >{\centering\arraybackslash}X 
  | >{\raggedleft\arraybackslash}X | }
 \hline
  \textbf{i} & \textbf{P\{I = i\}}\\
 \hline
 1  & 0.6\\
  \hline
 2  & 0.4\\
\hline
\end{tabularx} \\ \\
Table: Marginal Distribution of A \space \space \space \space \space \space \space \space \space \space \space \space \space \space Table: Marginal Distribution of I
\newpage \noindent
Now, we'll check for \textbf{(1)} \\ \\
P\{A = 1, I = 1\} = 0.18 = P\{A = 1\} . P\{I = 1\} = (0.3).(0.6) = 0.18 \longrightarrow \space \space satisfied\\ \\
P\{A = 1, I = 2\} = 0.12 = P\{A = 1\} . P\{I = 1\} = (0.3).(0.4) = 0.12 \longrightarrow \space \space satisfied\\ \\
P\{A = 2, I = 1\} = 0.30 = P\{A = 1\} . P\{I = 1\} = (0.5).(0.6) = 0.30 \longrightarrow \space \space satisfied\\ \\
P\{A = 2, I = 2\} = 0.20 = P\{A = 1\} . P\{I = 1\} = (0.5).(0.4) = 0.20 \longrightarrow \space \space satisfied\\ \\
P\{A = 3, I = 1\} = 0.12 = P\{A = 1\} . P\{I = 1\} = (0.2).(0.6) = 0.12 \longrightarrow \space \space satisfied\\ \\
P\{A = 3, I = 2\} = 0.08 = P\{A = 1\} . P\{I = 1\} = (0.2).(0.4) = 0.08 \longrightarrow \space \space satisfied\\ \\ \\

\noindent As seen above, \textbf{(1)} has been satisfied for all a and i. Hence, the snowy days in Ankara, \textbf{A}, and the snowy days in Istanbul, \textbf{I}, are \textbf{independent}.

\end{document}


