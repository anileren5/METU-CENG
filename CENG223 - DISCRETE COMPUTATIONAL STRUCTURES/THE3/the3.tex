\documentclass[11pt]{article}
\usepackage[utf8]{inputenc}
\usepackage[dvips]{graphicx}
\usepackage{fancybox}
\usepackage{verbatim}
\usepackage{array}
\usepackage{latexsym}
\usepackage{alltt}
\usepackage{hyperref}
\usepackage{textcomp}
\usepackage{color}
\usepackage{amsmath}
\usepackage{amsfonts}
\usepackage{tikz}
\usepackage{float}
\usepackage{amssymb}
\usepackage[hmargin=3cm,vmargin=5.0cm]{geometry}
%\topmargin=0cm
\topmargin=-2cm
\addtolength{\textheight}{6.5cm}
\addtolength{\textwidth}{2.0cm}
%\setlength{\leftmargin}{-5cm}
\setlength{\oddsidemargin}{0.0cm}
\setlength{\evensidemargin}{0.0cm}


\begin{document}

\section*{Student Information } 
%Write your full name and id number between the colon and newline
%Put one empty space character after colon and before newline
Full Name : Anıl Eren Göçer \\
Id Number : 2448397 \\


\section*{Question 1}
\textbf{Lemma:\space$((k \in Z^{+})\wedge (n \in Z^+)) \rightarrow (k^n \in Z^{+}).$ (The lemma can be used without giving a proof.)}
\newline \newline
\indent Assume that 1 is not the smallest positive integer, since 1 is positive, by the Well-Ordering Principle, there is a smallest positive integer, say k, and $k<1$. If we multiply the inequality \newline

\begin{center}
    $0 < k < 1$
\end{center}  \newline \newline
by k,then we get
\begin{center}
    $0 < k^2 < k$ \space \space .
\end{center} \newline \newline

By the \textbf{Lemma}, $k^2 \in Z^{+}$, implying that there is an element $k^2$ which is smaller than k and k is not the smallest integer. Therefore, our assumption has been contradicted. \newline  \newline 
\indent Hence, 1 is the smallest positive integer. 
\newpage
\section*{Question 2}
\noindent \textbf{First part of the proof:} \newline \newline
For $S(1,n)$: \newline \newline
Basis step: $S(1,1): x_1 = 1$, there is 1 solution which is $x_1 = 1$\newline
Also,$f(1,1) = \dfrac{(1+1-1)!}{1!.(1-1))} = \dfrac{1!}{1!.0!} = 1$. So,$S(1,1)$ is correct. \newline \newline
Inductive step: Assume $S(1,n)$ , i.e $x_1 = n$ has $f(1,n) = \dfrac{(n+1-1)!}{n!.(1-1)!} = 1$ solutions. \newline
Then, for $S(1,n+1)$, i.e $x_1 = n+1$ \newline By assumption, we should have 1
solution since we are just adding 1 on RHS. \newline
Also, $f(1,n+1) = \dfrac{(n+1+1-1)!}{(n+1)!(1-1)!} = \dfrac{(n+1)!}{(n+1)!} = 1$.\newline
$\therefore$ Hence , $S(1,n)$ is true. \newline \newline


\noindent For $S(m,1)$: \newline \newline
Basis step: $S(1,1): x_1 = 1$, there is 1 solution which is $x_1 = 1$\newline
Also,$f(1,1) = \dfrac{(1+1-1)!}{1!.(1-1))} = \dfrac{1!}{1!.0!} = 1$. So,$S(1,1)$ is correct. \newline \newline
Inductive step: Assume $S(m,1)$ , i.e $x_1 + x_2 + ....... + x_m = 1$ has 
$f(m,1) = \dfrac{(1+m-1)!}{1!.(m-1)!} = \dfrac{(m!)}{(m-1)!} = m$ solutions.\newline 
Then, for $S(m+1,1)$ i.e ,$x_1 + x_2 + ....... + x_m + x_{m+1}= 1$: \newline
There are two cases: $x_{m+1} = 0$ and $x_{m+1} = 1$ \newline \newline
For $x_{m+1} = 0$: the equation becomes $x_1 + x_2 + ....... + x_m = 1$ and by assumption there are m solutions. \newline
For $x_{m+1} = 1$: There is only 1 solution represented by (m+1)-tuples as (0,0,0,.....,1).\newline 
 \newline Therefore, total number of solutions is m + 1. \newline
Also, $f(m+1,1) = \dfrac{(1 + m + 1 - 1)!}{1!.(m+1-1)!} = \dfrac{(m+1)!}{m!} = m+1$ \newline
$\therefore$ Hence , $S(m,1)$ is true. \newline \newline \newline \newline

\noindent \textbf{Second part of the proof:} \newline \newline
\noindent  Assume $S(m,n+1)$ and $S(m+1,n)$ are true, i.e the numbers of solutions for \newline \newline

\noindent $x_1 + x_2 + .......... + x_m = n+1$\space \space \space \space \space \space \space \space \space \space \space (1) \newline \newline
$x_1 + x_2 + .......... + x_m + x_{m+1} = n$ \space \space \space \space \space (2) \newline \newline \newline
are $f(m,n+1) = \dfrac{(n+m)!}{(n+1)!.(m-1)!}$ and $f(m+1,n) = \dfrac{(n+m)!}{n!.m!}$ , respectively. \newline \newline


\noindent For $S(m+1,n+1)$: \newline \newline
the solutions of the equation \newline \newline
$x_1 + x_2 + .......... + x_m + x_{m+1} = n+1$ \space \space \space \space \space (3) \newline \newline
can be divided into two parts : $x_{m+1} = 0$ and $x_{m+1} > 0$ .
\noindent 
\newline \newline i) For $x_m+1 = 0$, equation (3) becomes equation (1), hence number of solutions is
\begin{center}
    $f(m,n+1) = \dfrac{(n+m)!}{(n+1)!.(m-1)!}$
\end{center}

\newline \newline 
\noindent \newline  ii)  For $x_m+1 > 0$, $x_{m+1}$ can be replaced by $x_{m+1}^\prime + 1$. It is guaranteed that $x_{m+1}^\prime \geqslant 0$ since $x_{m+1} > 0$. Therefore, we will not have any problem with restrictions in the question by doing this replacing. \newline \newline
Now, equation (3) becomes \newline \newline
$x_1 + x_2 + .......... + x_m + x_{m+1}^\prime = n$ , which is in the form consistent with (2)\newline \newline
so, the number of solution is 
\begin{center}
    $f(m+1,n) = \dfrac{(n+m)!}{n!.m!}$
\end{center}
Hence, \newline \newline
$\therefore$ The total number of solution is \newline
\begin{center}
    $\dfrac{(n+m)!}{(n+1)!.(m-1)!}$ + $\dfrac{(n+m)!}{n!.m!}$ = $\dfrac{(n+m)!.[(n+1)+m]}{(n+1)!.m!}$
\end{center}
\begin{center}
    $= \dfrac{[(n+1)+(m+1)-1]!}{(n+1)!.[(m+1)-1]!}$
\end{center}
Also $f(m+1,n+1) = \dfrac{[(n+1) + (m+1) -1 ]!}{(n+1)!.[(m+1)-1]!}$\newline
$\therefore$ $S(m+1,n+1)$ is also true. \newline \newline

\noindent Hence, we have proven $S(m,n)$ is true.

\newpage

\section*{Question 3}
\textbf{a.} \newline \newline
Count the number of 1 x 1 squares in the figure = 21 .\newline \newline
Each square can contain 4 triangles in the desired orientation and size: 21 x 4 = 84 .\newline \newline
On the diagonal of the half square we can have another 7 triangles.\newline \newline
Total number of triangles is 84 + 7 = 91 . \newline \newline
Hence, there are \textbf{91} triangles congruent to the one drawn in the figure, with
the same size and of any orientation. \newline \newline 
\textbf{b.} \newline \newline
\indent By \textbf{the Principle of Inclusion - Exclusion}, the number of functions from a set with 6 elements to a set with 4 elements is 
\begin{center}
    $4^6 - {4 \choose 1}.3^6 + {4 \choose 2}.2^6 - {4 \choose 3}.1^6 = 4916 - 2916 + 384 - 4 = 1560$\newline
\end{center}
\begin{center}
\end{center}



\newpage
\section*{Question 4}
\textbf{a.} \newline \newline 
\indent Let $a_{n}$ be the number of strings over the alphabet $\Sigma = \{0,1,2\}$ of length n that contain two consecutive symbols that are the same.\newline
Also, say \textbf{valid} string means a string over the alphabet $\Sigma = \{0,1,2\}$ of length n that contain two consecutive symbols that are the same. \newline \newline
There $a_{n-1}$ valid strings of length (n - 1). We can produce 3.$a_{n-1}$ valid strings of length n by placing any of $\{0,1,2\}$ at the end of each valid string of length (n - 1). \newline \newline
There are $3^{n-1}$ - $a_{n-1}$ non-valid strings of length (n - 1). If we put the $(n-1)th$ element again as the $nth$ element at the end of each non-valid strings of length (n - 1), we can produce a valid string of length n. So, we can produce $3^{n-1}$ - $a_{n-1}$ valid strings of length n. \newline \newline
Therefore, 
\begin{center}
    $a_n$ $=$  $3.a_{n-1}$ $+$ $3^{n-1}$ $-$ $a_{n-1}$
\end{center}
\begin{center}
    $a_n$ $=$ $2.a_{n-1}$ $+$ $3^{n-1}$
\end{center}
Hence, we obtained the above recurrence relation for the number of strings over the alphabet Σ = {0, 1, 2} of length n
that contain two consecutive symbols that are the same. \newline \newline
\newline \textbf{b.} \newline \newline
\indent Initial conditions for the recurrence relation are $a_1$ $=$ $0$ , $a_2$ $=$ $3$. \newline \newline
\textbf{c.} \newline \newline
\begin{center}
    $a_n$ $=$ $2.a_{n-1}$ $+$ $3^{n-1}$
\end{center}
Find homogeneous solution $a_n^h$ : \newline \newline
\indent Characteristic equation for the recurrence relation: $\alpha - 2$. So, $\alpha = 2$ is the characteristic root.\newline
Therefore, homogeneous solution $a_n^h$ is in the form of $A$ . $2^n$ where $A$ is a constant. 
\begin{center}
     $a_n^h$ = $A$ . $2^n$
\end{center}
Find particular solution $a_n^p$ : \newline \newline
\indent Non-homogeneity factor is $3^{n-1}$ , so particular solution $a_n^p$ is in the form of $B$ . $3^n$ where $B$ is a constant.
\begin{center}
     $a_n^p$ = $B$ . $3^n$
\end{center}
Now, find the constants $A$ and $B$ using initial conditions. \newpage

\begin{cases}
    $a_1$ $=$ $A$ . $2^1$ $+$ $B$ . $3^1$ $=$ $2A$ $+$ $3B$ $=$ $0$ \\
    $a_2$ $=$ $A$ . $2^2$ $+$ $B$ . $3^2$ $=$ $4A$ $+$ $9B$ $=$ $3$
\end{cases}
\rightarrow $\space \space \space \space A$ $ = $ $\dfrac{-3}{2}$ , $B$ $=$ $1$  . \newline \newline\newline
We have $a_n = a_n^h + a_n^p$ , \space  $a_n^h = $ $\dfrac{-3}{2}$ . $2^n$ , $a_n^p = $ $1$ . $3^n$\space \newline \newline

Thus,

\begin{center}
$a_n$ $=$ $\dfrac{-3}{2}$ . $2^n$ $+$ $1$ . $3^n$ $=$ $-3$ . $2^{n-1}$ $+$ $3^n$
\end{center}
Hence, by solving the recurrence relation we get 
\begin{center}
$a_n$ $=$ $-3$ . $2^{n-1}$ $+$ $3^n$
\end{center}
\end{document}


