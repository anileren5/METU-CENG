\documentclass[11pt]{article}
\usepackage[utf8]{inputenc}
\usepackage[dvips]{graphicx}
\usepackage{fancybox}
\usepackage{verbatim}
\usepackage{array}
\usepackage{latexsym}
\usepackage{alltt}
\usepackage{hyperref}
\usepackage{textcomp}
\usepackage{color}
\usepackage{amsmath}
\usepackage{amsfonts}
\usepackage{tikz}
\usepackage{fitch}  % to use fitch
\usepackage{float}
\usepackage[hmargin=3cm,vmargin=5.0cm]{geometry}
%\topmargin=0cm
\topmargin=-2cm
\addtolength{\textheight}{6.5cm}
\addtolength{\textwidth}{2.0cm}
%\setlength{\leftmargin}{-5cm}
\setlength{\oddsidemargin}{0.0cm}
\setlength{\evensidemargin}{0.0cm}


\begin{document}

\section*{Student Information } 
%Write your full name and id number between the colon and newline
%Put one empty space character after colon and before newline
Full Name :  Anıl Eren Göçer \\
Id Number :  2448397 \\

\section*{Question 1}
I am asked to prove the following set equivalence using membership notation and logical equivalences.
\begin{equation*}
    (A \cup B) \setminus (A \cap B) \equiv (A \setminus B) \cup (B \setminus A)
\end{equation*}

\noindent
\\\textbf{Steps and Justifications} \\\\
\begin{proof}
\fitchline{$(A \cup B) \setminus (A \cap B)$}{LHS of the equivalence}
\fitchline{$\equiv \{ x | x \in (A \cup B) \wedge x \notin (A \cap B) \}$}{Defn. of set difference}
\fitchline{$\equiv \{x|x \in (A \cup B) \wedge \neg(x \in (A \cap B))\}$}{Defn. of \notin} 
\fitchline{$\space \space \space \equiv \{x| (x \in A \vee x \in B ) \wedge \neg(x \in (A \cap B))\}$}{Defn. of union}
\fitchline{$\equiv \{x| (x \in A \vee x \in B) \wedge \neg (x \in A \wedge x \in B)\}$}{Defn. of intersection}
\fitchline{$\equiv \{x| (x \in A \vee x \in B) \wedge (\neg (x \in A) \vee \neg (x \in B))\}$}{De Morgan's Law for Propositional Logic}
\fitchline{$\equiv \{x| (x \in A \vee x \in B) \wedge (x \notin A \vee x \notin B)\}$}{Defn. of \notin}
\fitchline{$\space \space \space \equiv \{x|((x \in A \vee x\in B) \wedge x \notin A) \vee  ((x \in A \vee x\in B) \wedge x \notin B)\}$}{Distributive laws}
\fitchline{$\equiv \{x|(x \notin A \wedge (x \in A \vee x\in B)) \vee (x \notin B \wedge (x \in A \vee x\in B))\}$}{Commutative laws}
\fitchline{$\equiv \{x| ((x \notin A \wedge x \in A) \vee (x \notin A \wedge x \in B)) \vee ((x \notin B \wedge x \in A) \vee (x \notin B \wedge x \in B)) \}$}{Distributive laws}
\fitchline{$\equiv \{x| ((\neg(x \in A) \wedge x \in A) \vee (x \notin A \wedge x \in B)) \vee ((x \notin B \wedge x \in A) \vee (\neg(x \in B) \wedge x \in B ))\}$}{Defn. of \notin}
\fitchline{$\space \space \space \equiv \{x| (F \vee (x \notin A \wedge x \in B)) \vee ((x \notin B \wedge x \in A) \vee F)\}$}{Negation laws}
\fitchline{$\equiv\{x|(x \notin A \wedge x \in B) \vee (x \notin B \wedge x \in A)\}$}{Identity laws}
\fitchline{$\equiv\{x|(x \in B \wedge x \notin A) \vee (x \in A \wedge x \notin B)\}$}{Commutative laws}
\fitchline{$\equiv\{x|(x \in A \wedge x \notin B) \vee (x \in B \wedge x \notin A)\}$}{Commutative laws}
\fitchline{$\equiv \{x| (x \in (A \setminus B)) \vee (x \in (B \setminus A)) \}$}{Defn. of set difference}
\fitchline{$\equiv \{x| x \in ((A \setminus B) \cup (B \setminus A)) \}$}{Defn. of union}
\fitchline{$\equiv (A \setminus B) \cup (B \setminus A)$}{}
\\\\\\
By starting with LHS, I have obtained RHS, so the set equivalence has been proved.
\end{proof}


\newpage
\section*{A Lemma}
In this section I am going to prove a lemma in order to use it in the 2nd question.
\newline \newline
\noindent
\underline{Claim:} If A is an uncountably infinite set and B is a countably infinite set.
Then A $\setminus$ B is an uncountably infinite set.
\newline \newline
Let A be an uncountably set and B be a countably set.
\newline \newline
Assume that A $\setminus$ B is a countably infinite set. ( We will contradict.) \textbf{(1)}
\newline \newline
B is a countably infinite set, so it would mean that $(A \setminus B) \cup B$ is countably infinite set \textbf{(finite union of countable sets is clearly countable)}. But then A $\subseteq$ $(A \setminus B) \cup B$, so A is contained in a countable set, then A must be countable \textbf{(2)}
\newline\newline
We have obtained a contradiction by \textbf{(1)},\textbf{(2)}
\newline\newline
Therefore, our assumption has been contradicted. Hence, A $\setminus$ B is an uncountably infinite set.
\newline\newline\newline\newline
\noindent
\textbf{\underline{Lemma:}} If A is an uncountably infinite set and B is a countably infinite set, then A $\setminus$ B is an uncountably infinite set.


\newpage
\section*{Question 2}
\section*{Solution}
Let A = $\{f\mid f\subseteq \mathbb{N} \times \{0,1\}$\} \newline\newline
I will represent the mappings from $\mathbb{N}$ to $\{0,1\}$  \newline \newline
$f(1) = a_1$ \newline
$f(2) = a_2$ \newline
$f(3) = a_3$ \space \space \space \space \space \space \space \space \space \space \space \space \space \space \space \space \space \space \space \space as n-tuples of $(a_1,a_2,a_3,.......)$ where $a_i$ = 0 or $a_i$ = 1 for i = 1,2,3,......... \newline
. \newline
. \newline
. \newline\newline
\noindent
Assume that A, the set of all mappings defined from $\mathbb{N}$ to $\{0,1\}$ (plus empty set) is countable. Then there exists a 1-to-1 correspondence between
$\mathbb{N}$ and the set A. \newline \newline
Suppose we have it. \newline \newline
$1 \xrightarrow{} (a_1,a_2,a_3,....) = f_1$ \newline
$2 \xrightarrow{} (b_1,b_2,b_3,....) = f_2$ \newline
$3 \xrightarrow{} (c_1,c_2,c_3,....) = f_3$ \newline
. \newline
. \newline
. \newline\newline

\noindent
Now construct a mapping from $\mathbb{N}$ to $\{0,1\}$ that is missed by the
enumeration.  \newline \newline
$f(1) = x_1$ \newline
$f(2) = x_2$ \newline
$f(3) = x_3$ \newline
. \newline
. \newline
. \newline
represented as $f_x = (x_1,x_2,x_3,......)$
\newline \newline such that \newline 
$x_1 \neq a_1 \rightarrow f_1 \neq f_x$ \newline
$x_2 \neq b_2 \rightarrow f_2 \neq f_x$ \newline
$x_3 \neq c_3 \rightarrow f_3 \neq f_x$ \newline
. \newline
. \newline
. \newline \newline
By design $f_x \in A$ and it is missed by the enumeration. So, there does not exist an enumeration listing each mapping in A. Hence A is an uncountably infinite set. 
\newpage
\noindent
Let B = $\{f \mid f:\{0,1\} \xrightarrow{} \mathbb{N}$,$f$ is a function \}
\newline \newline
I will represent the functions defined from $\{0,1\}$ to $\mathbb{N}$ as such 2-tuples.

\newline \newline\noindent  \newline \newline
$f_a(0) = a_0$ \newline
$f_a(1) = a_1$  \space \space \rightarrow \space  \space $(a_0,a_1), a_0 \in \mathbb{N},a_1 \in \mathbb{N}$  \newline  \newline  \newline 
\noindent
By the following enumeration. \newline \newline
\noindent
(1,1), \space \space \space \space \space \space \space \space \space \space \space \space\space \space \space \space \space \space \space \space \space \space \space \spacespace \space \space \space \space \space \space \space \space  $\rightarrow$ \space \space \space $a_0 + a_1 = 2$   \space  $\rightarrow$ \space \space correspondence to (2-1) 1
\newline (1,2),(2,1),  \space \space \space \space \space \space \space \space \space \space \space \space\space \space \space \space \space \space \space \space \space \space \space \space    $\rightarrow$ \space \space \space $a_0 + a_1 = 3$ \space  $\rightarrow$ \space \space correspondence to (3-1) 2
\newline (1,3),(2,2),(2,1), \space \space \space \space \space \space \space \space \space \space \space \space\space \space \space \space \space   $\rightarrow$ \space \space \space $a_0 + a_1 = 4$ \space  $\rightarrow$ \space \space correspondence to (4-1) 3
\newline (1,4),(2,3),(3,2),(4,1), \space \space \space \space \space \space \space \space \space \space   $\rightarrow$ \space \space \space $a_0 + a_1 = 5$ \space  $\rightarrow$ \space \space correspondence to (5-1) 4
\newline
. \newline
. \newline
. \newline \newline \newline 
\noindent
There is an one-to-one correspondence between B and $\mathbb{N}$, we have an enumeration method listing all elements of B. Hence B is a countably infinite set. \newline \newline \newline 
\noindent
By the \textbf{\underline{Lemma}}, since A is an uncountably infinite set and
B is countably infinite set. The given set $\{f\mid f\subseteq \mathbb{N} \times \{0,1\}$\} $\setminus$
    $\{f \mid f:\{0,1\} \xrightarrow{} \mathbb{N}$,$f$ is a function \}, which is equal to $A \setminus B$ ,would be uncountably infinite set.





\newpage
\section*{Question 3}
Prove that the function $f(n) = 4^n +5n^2logn$ is not $O(2^n)$.
\section*{Solution}
Assume that  $f(n) = 4^n +5n^2logn$ is $O(2^n)$. (We will contradict.) \newline \newline
Then, there exists c and k constants such that \newline \newline
\indent\indent $4^n +5n^2logn < c.2^n $ ,\space  \space for all $n \geq k$ \newline \newline
\indent \indent $4^n/2^n + (5n^2logn)/2^n < c$ ,\space  \space for all $n \geq k$ \newline \newline
\indent \indent $2^n + (5n^2logn)/2^n < c$ ,\space  \space for all $n \geq k$ \newline \newline
\indent \indent $2^n < c,$ for all $n \geq k$ \newline \newline

\noindent
This cannot hold for all $n \geq k$, because as n goes to infinity LHS of the inequality goes to infinity while RHS of the inequality remains constant, we get a
contradiction. \newline \newline
The assumption has been contradicted.\newline \newline
Hence,$f(n) = 4^n +5n^2logn$ is not $O(2^n)$.

\newpage
\section*{Question 4}
Given two positive integers x and n such that $x > 2$ and $n > 2$, and the congruence relation \newline $(2x-1)^n - x^2 \equiv -x -1 (mod(x-1))$. I am required to determine the value of x. 
\section*{Solution}
 $(2x-1)^n - x^2 \equiv -x -1 (mod(x-1))$ \newline \newline
 $(2x-1)^n -x^2 + x + 1 \equiv 0 (mod(x-1))$ \newline \newline
 $(2x-1)^n(mod(x-1)) + (-x^2+x+1)(mod(x-1)) \equiv 0(mod(x-1))$ \newline \newline
 $[(2x-1)(mod(x-1))]^n(mod(x-1)) + (-x^2+x+1)(mod(x-1)) \equiv 0(mod(x-1))$ \newline \newline
 \indent \indent $2x-1 = 2(x-1) + 1 \rightarrow remainder = 1 \rightarrow 2x-1 \equiv 1 (mod(x-1))$ \newline \newline
 \indent \indent $-x^2+x+1 = -x.(x+1) + 1 \rightarrow remainder = 1 \rightarrow -x^2+x+1 \equiv 1(mod(x-1))$ \newline \newline
$1^n(mod(x-1)) + 1(mod(x-1)) \equiv 0(mod(x-1))$ \newline \newline
$1(mod(x-1)) + 1(mod(x-1)) \equiv 0(mod(x-1))$ \newline \newline
$2 (mod(x-1)) \equiv 0 (mod(x-1))$ \newline \newline
    \indent \indent $x-1 \mid 2 , x > 2$ \newline \newline
$x - 1 = 2$\newline \newline
$x = 3$
\noindent \newline \newline
The value of x is 3.
\end{document}

