\documentclass[11pt]{article}
\usepackage[utf8]{inputenc}
\usepackage[dvips]{graphicx}
\usepackage{fancybox}
\usepackage{verbatim}
\usepackage{array}
\usepackage{latexsym}
\usepackage{alltt}
\usepackage{hyperref}
\usepackage{textcomp}
\usepackage{color}
\usepackage{amsmath}
\usepackage{amsfonts}
\usepackage{tikz}
\usepackage{fitch}  % to use fitch
\usepackage{float}
\usepackage[hmargin=3cm,vmargin=5.0cm]{geometry}
\usetikzlibrary{shapes,arrows,positioning}
%\topmargin=0cm
\topmargin=-2cm
\addtolength{\textheight}{6.5cm}
\addtolength{\textwidth}{2.0cm}
%\setlength{\leftmargin}{-5cm}
\setlength{\oddsidemargin}{0.0cm}
\setlength{\evensidemargin}{0.0cm}


\begin{document}

\section*{Student Information } 
%Write your full name and id number between the colon and newline
%Put one empty space character after colon and before newline
Full Name : Anıl Eren Göçer \\
Id Number : 2448397  \\


\section*{Question 1}
\noindent Say $\sum\limits^{\infty}_{n=0}{a_n.x^n} = A(x)$ \newline \newline
Summing both side of the recurrence relation from n=1 to $n=\infty$, we obtain \newline \newline
$\sum\limits^{\infty}_{n=1}{a_n.x^n}$ = $\sum\limits^{\infty}_{n=1}{(a_{n-1} + 2^n).x^n}$ \newline \newline
$A(x) - a_0.x^0$ = $\sum\limits^{\infty}_{n=1}{a_{n-1}.x^n}$ + $\sum\limits^{\infty}_{n=1}{2^n.x^n}$\newline \newline
$A(x) - 1 = $ $x.\sum\limits^{\infty}_{n=1}{a_{n-1}.x^{n-1}}$ + $\sum\limits^{\infty}_{n=1}{2^n.x^n}$ \newline \newline
$A(x) - 1 = $ $x.\sum\limits^{\infty}_{n=0}{a_n.x^n}$ + $\sum\limits^{\infty}_{n=1}{2^n.x^n}$\newline \newline
$A(x) - 1 = $ $x.A(x)$ + $((\sum\limits^{\infty}_{n=0}{2^n.x^n}) - 2^0.x^0)$ \newline \newline
$A(x) - 1 = $ $x.A(x)$ + $((\sum\limits^{\infty}_{n=0}{2^n.x^n}) - 1)$ \newline \newline
$A(x) - 1 = $ $x.A(x)$ + $(\dfrac{1}{1-2x} - 1)$ \newline \newline
$A(x) - x.A(x) = \dfrac{1}{1-2x}$ \space \space \space $\xrightarrow{}$ \space \space $A(x)(1-x) = \dfrac{1}{1-2x}$ \space \space \space $\xrightarrow{}$ \space \space $A(x) = \dfrac{1}{(1-x).(1-2x)}$ \newline \newline
$A(x) = (-1).\dfrac{1}{1-x} + (2).\dfrac{1}{1-2x}$ \space \space \space  \textbf{(By partial fractions)} \newline \newline
$(1^0,1^1,1^2,..........,1^n,...)$\space = \space $\dfrac{1}{1-x}$ \space \space \space $\space \space \space \Longrightarrow \space \space ((-1).1^0,(-1).1^1,(-1).^2,..........,(-1).1^n,...)$ \space = \space  $\space (-1).\dfrac{1}{1-x}$ \newline \newline
$(2^0,2^1,2^2,..........,2^n,...)$ \space \space = \space $\dfrac{1}{1-2x}  \space \space$\space \Longrightarrow \space  $\space \space(2.2^0,2.2^1,2.^2,..........,2.2^n,...)$ = $(2)\space.\space\dfrac{1}{1-2x}$ \newline

\noindent $A(x) = (-1).\dfrac{1}{1-x} + (2).\dfrac{1}{1-2x} = ((-1).1^0,(-1).1^1,(-1).^2,.......,(-1).1^n,...)$ \space + \space$(2.2^0,2.2^1,2.^2,.......,2.2^n,...)$ \newline \newline
$A(x) = \sum\limits^{\infty}_{n=0}{a_n.x^n} = (2.2^0 - 1^0,2.2^1 - 1^1,2.2^2 - 1^2,........,2.2^n - 1^n,...)$ \newline \newline
Hence, $a_n = 2.2^n - 1^n = 2^{n+1} - 1. \newline \newline \newline \newline $
\noindent
The recurrence relation has been solved as \textbf{$a_n = 2^{n+1} - 1$}.


\newpage
\section*{Question 2}
$R = \{(1,1),(1,2),(1,3),(1,9),(1,18),(2,2),(2,18),(3,3),(3,9),(3,18),(9,9),(9,18),(18,18)\}$ \newline \newline
\textbf{a)} Hasse Diagram of $R$ \newline \newline

\begin{center}
\begin{tikzpicture}
\node [circle,draw = black] (v1) at (5,0) {18};
\node [circle,draw = black] (v2) at (3,-2) {2};
\node [circle,draw = black] (v3) at (7,-2) {9};
\node [circle,draw = black] (v4) at (7,-4) {3};
\node [circle,draw = black] (v5) at (5,-6) {1};
\draw[] (v1)--(v2);
\draw[] (v1)--(v3);
\draw[] (v2)--(v5);
\draw[] (v3)--(v4);
\draw[] (v4)--(v5);
\end{tikzpicture}
\end{center}
\newline  \newline
\noindent \textbf{b)} Matrix Representation of R \newline \newline
\[
    M_R = 
    \begin{pmatrix}
        1 & 1 & 1 & 1 & 1 \\
        0 & 1 & 0 & 0 & 1 \\
        0 & 0 & 1 & 1 & 1 \\
        0 & 0 & 0 & 1 & 1 \\
        0 & 0 & 0 & 0 & 1 \\ 
    \end{pmatrix}
    
\]
Above matrix is representation of the relation $aRb$ where $a$ is represented by rows and $b$ is represented by columns. Also, note that the columns are for \{1,2,3,9,18\} from left to right and the rows are for \{1,2,3,9,18\}.
\newline \newline
\newpage
\noindent \textbf{c)} \newline \newline
Yes, $(A,R)$ is a lattice. \newline \newline
Explanation: \newline \newline
We know that a partial order relation is a \textbf{lattice} if for every pair of elements there is a unique Least Upper Bound ($LUB$) and unique Greatest Lower Bound ($GLB$). \newline \newline 

 \noindent$LUB(1,2) = 2$ \space \space  \space   \space \space  \space   \space \space  \space  \space \space  \space$GLB(1,2) = 1$ \newline

\noindent$LUB(1,3) = 3 $ \space \space  \space   \space \space  \space   \space \space  \space  \space \space  \space$GLB(1,3) = 1$ \newline

\noindent$LUB(1,9) = 9$ \space \space  \space   \space \space  \space   \space \space  \space  \space \space  \space$GLB(1,9) =1 $ \newline

\noindent$LUB(1,18) = 18$ \space \space  \space   \space \space  \space   \space \space  \space  \space \space $GLB(1,18) = 1$ \newline

\noindent$LUB(2,3) = 18$ \space \space  \space   \space \space  \space   \space \space  \space  \space \space $GLB(2,3) = 1$ \newline

\noindent$LUB(2,9) = 18$ \space \space  \space   \space \space  \space   \space \space  \space  \space \space $GLB(2,9) = 1$ \newline

\noindent$LUB(2,18) = 18$ \space \space  \space   \space \space  \space   \space \space  \space  \space \space $GLB(2,18) = 2$ \newline

\noindent$LUB(3,9) = 9 $ \space \space  \space   \space \space  \space   \space \space  \space  \space \space $GLB(3,9) = 3$ \newline

\noindent$LUB(3,18) = 18 $ \space \space  \space   \space \space  \space   \space \space  \space  \space \space $GLB(3,18) = 3 $ \newline

\noindent$LUB(9,18) = 18 $ \space \space  \space   \space \space  \space   \space \space  \space  \space \space $GLB(9,18) = 9$ \newline \newline

\noindent So, for every pair of elements there is a unique Least Upper Bound ($LUB$) and unique Greatest Lower Bound ($GLB$). \newline \newline
Hence, $(A,R)$ is a lattice. 















\newpage
\noindent \textbf{d)} \newline \newline
$R = \{(1,1),(1,2),(1,3),(1,9),(1,18),(2,2),(2,18),(3,3),(3,9),(3,18),(9,9),(9,18),(18,18)\}$ \newline \newline
\noindent
$S = \{(1,1),(2,1),(3,1),(9,1),(18,1),(2,2),(18,2),(3,3),(9,3),(18,3),(9,9),(18,9),(18,18)\}$ \newline \newline
$R_s = R$ $\cup$ $S$ = $\{(1,1),(1,2),(1,3),(1,9),(1,18),(2,2),(2,18),(2,1),(3
,3),(3,9),(3,18),(3,1),\newline  (9,3),(9,9),(9,18),(9,1),(18,18),(18,3),(18,2),(18,9),(18,1)\}$ \newline \newline
\begin{center}
    Matrix representation of symmetric closure $R_s$ of $R$ 
\end{center}
\[
    M_R_s = 
    \begin{pmatrix}
        1 & 1 & 1 & 1 & 1 \\
        1 & 1 & 0 & 0 & 1 \\
        1 & 0 & 1 & 1 & 1 \\
        1 & 0 & 1 & 1 & 1 \\
        1 & 1 & 1 & 1 & 1 \\ 
    \end{pmatrix}
    
\]

Above matrix is representation of the symmetric closure $aR_sb$ where $a$ is represented by rows and $b$ is represented by columns. Also, note that the columns are for \{1,2,3,9,18\} from left to right and the rows are for \{1,2,3,9,18\}.


\noindent \newline \newline \textbf{e)}
\newline \newline i) 2 and 9 are not comparable because $2\not\mathrel{R}9$ and $9\not\mathrel{R}2$ .\newline \newline
ii) 3 and 18 are comparable because $18\not\mathrel{R}3$ but $3$ $R$ $18$ .


\newpage
\section*{Question 3}
\textbf{a)} \newline \newline
For a relation $R$ on $A$ to be anti-symmetric, $\forall x,y \in A $ $\space (xRy \wedge yRx \xrightarrow{} x = y )$ must be satisfied (i.e if $xRy$ and $x$ and $y$ are distinct, then $y$$\not \mathrel{R}$$x$, for any (x,y) ordered pair).  \newline \newline

\noindent For the entries corresponding to ordered pairs that are located on the diagonal of the matrix representation of relation R on A (i.e (x,y) pairs such that x = y )  \newline

\noindent There are two possible choices for each entry (pair): i) $xRx$ \space,\space ii) $x$ $\not\mathrel{R}$$x$ \newline
And, there are n such entries \newline \newline
So,we have $2^n$ different choices for these entries (pairs). \newline \newline \newline 

\noindent For the entries corresponding to ordered pairs which are \textbf{not} located on the diagonal of the matrix representation of relation $R$ on $A$ (i.e (x,y) pairs such that $x \neq y$) .
\newline \newline
There are three possible choices for each entry (pair) : \newline \newline
i) $xRy$ and $yRx$ \newline
ii) $x\not\mathrel{R}y$ and $yRx$ \newline
iii) $x\not\mathrel{R}y$ and $y\not\mathrel{R}x$ \newline \newline

\noindent And, there are ${n \choose 2}$ = $\dfrac{n.(n-1)}{2}$ such pairs \newline
\newline So, we have $3^{n \choose 2}$ = $3^{\dfrac{n.(n-1)}{2}}$ different choices for these entries (pairs) . \newline \newline \newline


\noindent Hence, there are $2^n.3^{n \choose 2}$ = $2^n.3^{\dfrac{n.(n-1)}{2}}$  different anti-symmetric binary relations R on A. \newline \newline


\noindent Answer =  $2^n.3^{n \choose 2}$ = $2^n.3^{\dfrac{n.(n-1)}{2}}$

\newpage
\noindent \textbf{b)} \newline \newline
For a relation $R$ on $A$ to be reflexive, $\forall x \in A$ $xRx$ must be satisfied.  \newline \newline
For a relation $R$ on $A$ to be anti-symmetric, $\forall x,y \in A $ $\space (xRy \wedge yRx \xrightarrow{} x = y )$ must be satisfied (i.e if $xRy$ and $x$ and $y$ are distinct, then $y$$\not \mathrel{R}$$x$, for any x,y ordered pair).  \newline \newline

\noindent For the entries corresponding to ordered pairs which are located on the diagonal of the matrix representation of relation $R$ ( i.e (x,y) ordered pairs such that x = y ) \newline

\noindent We want R to be anti-symmetric, then there are two possible choices: i) $xRx$ , ii) $x\not \mathrel{R}x$
\newline \newline
But, we want also R to be reflexive, so \textbf{only 1} choice would be: $xRx$ \newline \newline
\noindent So, we have 1 choice for each pair. There are n such pairs. Hence there are $1^n = 1$ choices for these pairs. \newline \newline

\noindent For the entries corresponding to ordered pairs which are \textbf{not} located on the diagonal of the matrix representation of relation $R$ on $A$ (i.e (x,y) pairs such that $x \neq y$) \newline \newline

\noindent There are three possible choices for each entry (pair) : \newline \newline
i) $xRy$ and $yRx$ \newline
ii) $x\not\mathrel{R}y$ and $yRx$ \newline
iii) $x\not\mathrel{R}y$ and $y\not\mathrel{R}x$ \newline \newline

\noindent And, there are ${n \choose 2}$ = $\dfrac{n.(n-1)}{2}$ such pairs \newline
\newline So, we have $3^{n \choose 2}$ = $3^{\dfrac{n.(n-1)}{2}}$ different choices for these entries (pairs) . \newline \newline \newline

\noindent Hence, there are $1$ . $3^{n \choose 2} = 3^{n \choose 2} = 3^{\dfrac{n.(n-1)}{2}}$ relations that are both reflexive and anti-symmetric on A. \newline \newline \newline


\noindent Answer = $3^{n \choose 2} = 3^{n \choose 2} = 3^{\dfrac{n.(n-1)}{2}}$. 
 
\end{document}