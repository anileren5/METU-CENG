\documentclass[12pt,a4paper, margin=1in]{article}
\usepackage{fullpage}
\usepackage{amsfonts, amsmath, pifont}
\usepackage{amsthm}
\usepackage{amsmath}

\usepackage{geometry}
 \geometry{
 a4paper,
 % total={210mm,297mm},
 left=10mm,
 right=10mm,
 top=5mm,
 bottom=10mm,
 }
 \author{
  Göçer, Anıl Eren\\
  \texttt{e2448397@ceng.metu.edu.tr}
}
\title{CENG 382 - Analysis of Dynamic Systems \\
20221\\
Take Home Exam 3 Solutions}
\begin{document}
\maketitle

\noindent\rule{19cm}{1.2pt}

\begin{enumerate}
% Write your solutions in the following items.

    \item % Q1
        \begin{enumerate}
            \item First, calculate the Jacobian matrix: \\
            \begin{center}
                $Df = \begin{bmatrix}
                    \dfrac{\partial f_1}{\partial x_1} & \dfrac{\partial f_1}{\partial x_2} \\ \\
                    \dfrac{\partial f_2}{\partial x_1} & \dfrac{\partial f_2}{\partial x_2}
                \end{bmatrix} = \begin{bmatrix}
                    -1 & 2x^2 \\
                    6x_1 & -2
                \end{bmatrix}$
            \end{center}

            for $\Tilde{x} = \begin{bmatrix}
                0 \\ 0
            \end{bmatrix}$;
            \begin{center}
                $Df(\Tilde{x}) = \begin{bmatrix}
                    -1 & 0 \\
                    0 & -2
                \end{bmatrix}$
            \end{center}

            Now, calculate eigenvalues: \\
            \begin{center}
                $\begin{vmatrix}
                    \lambda + 1 & 0 \\
                    0 & \lambda + 2
                \end{vmatrix} = (\lambda + 1 )(\lambda + 2)=0$ \space \space $\longrightarrow $ \space \space $\lambda_1 = -1, \lambda_2 = -2$
            \end{center}

            This is a continuous time system and all eigenvalues, $\lambda_1 = -1$ and $\lambda_2 = -2$, have negative real parts, so the fixed point $\Tilde{x} = \begin{bmatrix}
                0 \\ 0
            \end{bmatrix}$ is \textbf{stable}. 

            
            \item (i) $V(x_1, x_2) = \dfrac{x_1^2}{2} + \dfrac{x_2^2}{4}$ is a polynomial function, so $V$ is continuous on $R^2$. Also, $V$ has continuous first partial derivatives, since it is a polynomial.
            In particular, $\dfrac{\partial V}{\partial x_1} = x_1$ and $\dfrac{\partial V}{\partial x_2} = \dfrac{x_2}{2}$ are the first partial derivatives of $V$. Since they are also polynomial, they are continuous.  \\

            (ii) $V(x_1, x_2) = \dfrac{x_1^2}{2} + \dfrac{x_2^2}{4}$ contains the terms $x_1^2$, $x_2^2$. These will be minimum if $x_1 = 0$, $x_2 = 0$. Thus, $V$ gets its unique minimum at the fixed point $\Tilde{x} = \begin{bmatrix}
                0 \\ 0
            \end{bmatrix}$. 
            
            \newpage

            (iii) Finally, we check the difference function $V' = \nabla V(x) . f(x)$ and examine if it satisfies
            $V' \leq 0$ for each x. 

            \begin{center}
            \begin{equation*}
            \begin{split}
                    V' & = \nabla V(x) . f(x) \\
                        & = \dfrac{\partial V}{\partial x_1}x_1' + \dfrac{\partial V}{\partial x_2}x_2' \\
                        & = x_1 (-x_1 + x_2^2) + \dfrac{x_2}{2}(-2x_2 + 3x_1^2) \\
                        & = -x_1^2 + x_1x_2^2 - x_2^2 + \dfrac{3}{2}x_2x_1^2 \\
                        & = -x_1^2(1-\dfrac{3}{2}x_2) - x_2^2(1-x_1)
            \end{split}
            \end{equation*}    
            \end{center}

            $V'$ is not less than or equal to 0 for all $(x_1,x_2) \in R^2$. However, we still have a chance to show stability.
\\

            Notice that $V' < 0$ for all $(x_1,x_2)$ such that $x_1 < 1$ and $x_2 < \dfrac{2}{3}$.  We can find such a region including the fixed point $\Tilde{x}$:
\\ \\
            For example; \\ \\
            Observe that level curves of the lyapunov function $V$ are the ellipses centered at origin with the axes
            $2\sqrt{\alpha}$ and $\sqrt{2\alpha}$. We can use these ellipses by satisfying the following conditions:
            \\\\

             $2\sqrt{\alpha} < \dfrac{2}{3} \longrightarrow \alpha < \dfrac{1}{9}$ \\ \\ 
             $\sqrt{2\alpha} < 1 \longrightarrow \alpha < \dfrac{1}{2}$

            These two inequalities gives the fact that  $\alpha < \dfrac{1}{9}$. \\\\

            So, we found the region as an ellipse:
            \begin{center}
                $\dfrac{x_1^2}{2} + \dfrac{x_2^2}{4} = \dfrac{1}{9}$
            \end{center}

            This region includes the fixed point and satisfies the conditions $x_1 < 1$ and $x_2 < \dfrac{2}{3}$ meaning $V' < 0$.  \\
            
            We can take the largest circle inside the ellipse as the attraction of region by setting the radius to the half of the minor axis of the ellipse: 
            \begin{center}
                $x_1^2 + x_2^2 = \dfrac{2}{9}$
            \end{center}
            
            
            Thus, if the system starts near the the fixed point (origin) in this circle, it must tend to go to the fixed point. \\ \\
            Thus, the fixed point $\Tilde{x} = \begin{bmatrix}
                0 \\ 0
            \end{bmatrix}$ is \textbf{stable}. 
    
        \end{enumerate}
        

\newpage



        
    \item % Q2
        I will use the function $V(x_1, x_2, x_3) = 4(x_1^2 + x_2^2 + x_3^3)$. \\

        (i) $V$ is a polynomial function, so it is continuous. \\ \\
        (ii) $V$ contains terms $x_1^2, x_2^2$ and $x_3^2$. These will be minimum if $x_i = 0$ for $i = 1, 2, 3$. Thus, $V$ has a  unique minimum at the fixed point $\Tilde{x} = \begin{bmatrix}
            0 \\ 0 \\ 0
        \end{bmatrix}$. \\
        
        (iii) Now, we will compare $V(x_1(k+1), x_2(k+1), x_3(k+1))$ and  $V(x_1(k), x_2(k), x_3(k))$. In other words, we will calculate the difference function
        \begin{center}
            $\Delta V = V(f(x)) - V(x) = V(x_1(k+1), x_2(k+1), x_3(k+1)) - V(x_1(k), x_2(k), x_3(k))$
        \end{center}
        and check if it satisfies $\Delta V \leq 0$ for all x. 

        \begin{center}
            $V(x_1(k), x_2(k), x_3(k)) = 4x_1^2(k) + 4x_2^2(k) + 4x_3^2(k)$ 
        \end{center}

            \begin{center}
            \begin{equation*}
            \begin{split}
                    V(x_1(k+1), x_2(k+1), x_3(k+1)) & = 4 (\dfrac{1}{2}x_1(k) + \dfrac{1}{2}x_2(k))^2 + 4(\dfrac{1}{2}x_3(k))^2 + 4 (\dfrac{1}{2}x_1(k) - \dfrac{1}{2}x_2(k))^2 \\ 
                    & = x_1^2(k) + x_1(k)x_2(k) + x_3^2(k) + x_1^2(k) - x_1(k)x_2(k) + x_3^2(k) \\
                    & = 2x_1^2(k) + 2x_2^2(k) + x_3^2(k)
            \end{split}
            \end{equation*}    
            \end{center}

        Then,

            \begin{center}
            \begin{equation*}
            \begin{split}
                    \Delta V & = (2x_1^2(k) + 2x_2^2(k) + x_3^2(k)) - (4x_1^2(k) + 4x_2^2(k) + 4x_3^2(k))\\
                    & = -2x_1^2(k) -2x_2^2(k) -3x_3^2(k)
            \end{split}
            \end{equation*}    
            \end{center}
        
            Since all square terms have negative sign, $\Delta V \leq 0$ for all $x \in R^3$.  \\

            This means that $V$ is not increasing along trajectories.  \\

            Hence, the fixed point $\Tilde{x} = \begin{bmatrix}
                0 \\ 0 \\ 0
            \end{bmatrix}$ is \textbf{stable} by Lyapunov Theorem. 

\newpage
    \item % Q3
        Find the fixed points of the system: 
        \begin{center}
            $x_1'(t) = x_1 + x_2 -4x_1(x_1^2 + x_2^2) = 0$ \\
            $x_2'(t) = -x_1 + x_2 -4x_2(x_1^2 + x_2^2) = 0$ \\ 
        \end{center}
            Multiply the first equation by $x_2$ and multiply the second equation by $x_1$:
        \begin{center}
            $x_1x_2 + x_2^2 -4x_1x_2(x_1^2 + x_2^2) = 0$ \\
            $-x_1^2 + x_1x_2 -4x_1x_2(x_1^2 + x_2^2) = 0$
        \end{center}
        Subtract the second equation from the first equation:
        \begin{center}
            $x_1^2 + x_2^2 = 0$
        \end{center}
        This gives us that $\Tilde{x} = \begin{bmatrix}0 \\ 0 \end{bmatrix}$ is \textbf{the only fixed point} of the system. \\\\

        Now, apply linearization around the fixed point $\Tilde{x} = \begin{bmatrix} 0 \\ 0\end{bmatrix}$: \\
        Calculate the Jacobian matrix:
        \begin{center}
            $Df = \begin{bmatrix}
                    \dfrac{\partial f_1}{\partial x_1} & \dfrac{\partial f_1}{\partial x_2} \\ \\
                    \dfrac{\partial f_2}{\partial x_1} & \dfrac{\partial f_2}{\partial x_2}
                \end{bmatrix} = \begin{bmatrix}
                    1 - 12x_1^2 -4x_2^2 & 1 -8x_1x_2 \\ \\
                    -1 - 8x_1x_2 & 1 - 4x_1^2 - 12x_2^2
                \end{bmatrix}$ \\
            $Df(\Tilde{x}) = \begin{bmatrix}
                1 & 1 \\
                -1 & 1
            \end{bmatrix}$
        \end{center}

        Calculate the eigenvalues: \\ \\
        $\begin{vmatrix}
            \lambda - 1 & -1 \\
            1 & \lambda -1
        \end{vmatrix} = 0$ $\longrightarrow (\lambda - 1)^2 + 1 = 0 \Longrightarrow \lambda_1 = 1 + i, \lambda_2 = 1 - i$
        \\\\
        All eigenvalues have positive real part. It is enough to have one eigenvalue with positive real part in order to say that a fixed point is unstable. So, $\Tilde{x} = \begin{bmatrix}
            0 \\0
        \end{bmatrix}$ is \textbf{unstable}. \\

        This means that \textbf{the system does not converge to a fixed point}.
        \\

        We know that the fixed point is unstable, but let's try $V(x_1, x_2) = \dfrac{x_1^2}{2} + \dfrac{x_1^2}{2}$ (it cannot be a Lyapunov function) .
        \\
        
        i) $V$ is a polynomial function, so it is continuous. Its first partial derivatives are also polynomial. So, it has continuous first partial derivatives.  \\

        ii) It contains square terms $x_1^2, x_2^2$ which get their minimum at 0. Therefore, $V$ gets its unique minimum at $\Tilde{x} = \begin{bmatrix}
            0 \\ 0
        \end{bmatrix}$

        \newpage

        iii) Now check $V'$:
        
            \begin{center}
            \begin{equation*}
            \begin{split}
                V' & = \dfrac{\partial V}{\partial x_1} x_1' +  \dfrac{\partial V}{\partial x_2} x_2' \\
                & = x_1(x_1 + x_2 +4x_1^3 - 4x_1x_2^2) + x_2(-x_1 + x_2 - 4x_2x_1^2 - 4x_2^3) \\
                & = x_1^2 + x_1x_2 -4x_1^4 -4x_1^2x_2^2 - x_1x_2 + x_2^2 -4x_2^2x_1^2 -4x_2^4 \\ 
                & = x_1^2 + x_2^2 -4x_1^4 -4x_2^4 -8x_1x_2 \\
                & = x_1^2 + x_2^2 -4(x_1^4 + 2x_1^2x_2^2 + x_2^4) \\
                & = x_1^2 + x_2^2 -4(x_1^2 + x_2^2)^2 \\
                & = (x_1^2 + x_2^2)[1 - 4 (x_1^2 + x_2^2)]
            \end{split}
            \end{equation*}    
            \end{center}

            We want $V' < 0$. However, we see that when $x_1^2 + x_2^2 < \dfrac{1}{4} = (\dfrac{1}{2})^2$, then $V'$ is actually positive. On the other hand, if  $x_1^2 + x_2^2 > \dfrac{1}{4} = (\dfrac{1}{2})^2$, then $V' < 0$. In other words, if $ x_1$ and $x_2$ are large enough, then the system is heading back toward origin by getting closer to the region bounded by the circle $x_1^2 + x_2^2  = (\dfrac{1}{2})^2$. 
            As a result, we see that \textbf{the system does not diverge to infinity}. \\ \\

            We found that the only fixed point, origin $\Tilde{x} = \begin{bmatrix}
                0 \\0
            \end{bmatrix}$ is unstable, and divergent behavior is impossible. What's left ? The Poincare-Bendixson theorem leaves us only one possible behavior: As $t \longrightarrow \infty$, we must have \textbf{$x(t)$ tending to a periodic orbit.} 
            And \textbf{this periodic orbit is the limit cycle we found above}. \\\\

            Hence, \textbf{the system has a periodic limit cycle  $x_1^2 + x_2^2  = (\dfrac{1}{2})^2$}.
    
        

\newpage

    \item % Q4 
        \begin{enumerate}
            \item To find fixed points of the system, we need to solve $\Tilde{x} = f(\Tilde{x})$:
                \begin{center}
                    $\Tilde{x} = 3 - \Tilde{x}^2 \longrightarrow \Tilde{x}^2 + \Tilde{x} - 3 = 0 \longrightarrow \Tilde{x_1} = \dfrac{-1 + \sqrt{13}}{2}, \Tilde{x_2} = \dfrac{-1 - \sqrt{13}}{2}$
                \end{center}
            The fixed points are $\Tilde{x_1} = \dfrac{-1 + \sqrt{13}}{2}$ and $\Tilde{x_2} = \dfrac{-1 - \sqrt{13}}{2}$.  \\
     
            
            \item We need to solve $f^2(x) = f(f(x)) = x$ where $f(x) = 3 - x^2$:

            \begin{center}
                $3 - (3 - x^2)^2 = x \longrightarrow 3 - x^4 - 9 + 6x^2 = x$ \\ $\longrightarrow x^4 - 6x^2 + x + 6 = 0$ \\
                $\longrightarrow (x+1)(x-2)(x^2 + x - 3) = 0$
            \end{center}

            I have found the roots of the equation as $x_1 = \dfrac{-1 + \sqrt{13}}{2}$, $x_2 = \dfrac{-1 - \sqrt{13}}{2}$, $x_3 = -1$ \\ and $x_4 = 2$. \\ 

            Now, we know the periodic points. Let's have a closer look at them: \\

            If we plug  $x_1 = \dfrac{-1 + \sqrt{13}}{2}$ or $x_2 = \dfrac{-1 - \sqrt{13}}{2}$ into the system, we always get the same state as the previous step. Actually, this happens because $x_1$ and $x_2$ are fixed points of the system. This means that $x_1$ and $x_2$ have prime periods of 1.  \\\\

            If we plug $x_3 = -1$ into the system, we observe:
            
            \begin{center}
            \begin{equation*}
            \begin{split}
                    3 - x_3^2 & = 3 - (-1)^2  = 2  = x_4 \\
                    3 - x_4^2 & = 3 - 2^2  = -1  = x_3 \\
                    3 - x_3^2 & = 3 - (-1)^2  = 2  = x_4 \\
                              & . \\
                              & . \\
                              & .
            \end{split}
            \end{equation*}    
            \end{center}

            We see that, when we plug $x_3$ or $x_4$ into the system, state will oscillate between $x_3$ and $x_4$. \\ This means that $x_3 = -1$ and $x_4 = 2$ are \textbf{periodic points of prime period 2.} \\

            Remember that we solved the equation $x = f(x)$ to find fixed points which are nothing but periodic points with prime period 1.  Moreover, remember that we solved the equation \\ $f^2(x) = f(f(x)) = x$ to find periodic points with period up to 2. And from these, I concluded that the relation between the fixed points and the periodic points of prime period 2 in terms of the equations that I use to calculate them is the following: \\

            \{ periodic points with prime period 2\} = \{roots of $f^2(x) = f(f(x)) = x$\} - \{roots of $x = f(x)$\} \\

            In particular, if we apply polynomial division and solve the equation
            \begin{center}
                $\dfrac{f^2(x) - x}{f(x) - x} = 0$
            \end{center}
            we can find the periodic points with prime period 2.
            
            
\newpage    
            \item Periodic points with prime period 2 are $x_3 = -1$ and $x_4 = 2$ which we found in part b.

            We should compute $(f^2)'(x_3)$ and $(f^2)'(x_4)$ to determine stability of them. \\ \\
            By chain rule: 
            \begin{center}
                $(f^2)'(x) = \dfrac{dff^2(x)}{dx} = f'(f(x)).f'(x)$
            \end{center}
            where $f(x) = 3 - x^2$ and $f'(x) = - 2x$ . \\ \\ So,

            \begin{center}
            \begin{equation*}
            \begin{split}
                 (f^2)'(x) & = -2 (3-x^2)(-2x) \\
                 & = -4x^3 + 12x
            \end{split}
            \end{equation*}    
            \end{center}

            By plugging $x_3 = -1$ and $x_4 = 2$:
            \begin{center}
                $(f^2)'(x_3 = -1) = -4.(-1)^3 + 12.(-1) = -8$ \\ 
                $(f^2)'(x_4 = 2) = -4.(2)^3 + 12.(2) = -8$ \\      
            \end{center}

            We see that both $(f^2)'(x_3)$ and  $(f^2)'(x_4)$ have absolute value greater than 1 i.e. $|(f^2)'(x_3)| > 1$ and $|(f^2)'(x_4)| > 1$. So, the system do not tend to these points. \\

            Therefore, both $x_3 = -1$ and $x_4 = 2$ are \textbf{unstable}. 
            
        \end{enumerate}


\end{enumerate}

\end{document}
